% Resumo em língua vernácula
\begin{center}
	{\Large{\textbf{Cloud Computing Assisted Smart Surveillance Based Safe Transportation System to Improve Crime Assistance on Smart Cities}}}
\end{center}

\vspace{0.5cm}

\begin{flushright}
	Autor: Hugo Barros Camboim\\
	Supervisor: Prof. Dr. Augusto José Venâncio Neto
\end{flushright}

\vspace{0.5cm}

\begin{center}
	\Large{\textsc{\textbf{Abstract}}}
\end{center}

\noindent Smart video surveillance plays a key role in offering technological assistance in public safety scenarios, due to its potential to allow events and objects to be automatically detected in real-time. The Smart video surveillance technology has attracted special attention in the field of public safety, particularly  in vehicular environments by the ability in assisting mission critical scenarios through real-time video processing based automatic detections (e.g., assaults, kidnaps, acts of violence etc.). Hence, it is possible to enable a tradeoff to be made between reactive and pro-active authorities when taking action, seeking to keep customized alert systems, carry out more efficient planning, and thus provide society with a better quality of life. Moreover, the association to Cloud Computing capabilities, is likely to provide advanced perspectives to smart surveillance applications, in terms to provisioning enhanced: (i) processing, to allow for deploying smart computing techniques for improved accuracy; (ii) storage, to adopt a wide variety of templates; (iii), and ubiquitous access, to keep connected everywhere and to any device. The investigation carried in this work focus in deploying a smart surveillance driven cloud-enabled approach to allow for event-based mobile applications in the public safety field. The objective is to allow optimized mobile applications, by which exploiting event-based scheme to provide to low resource consumption (energy, CPU and network usage) perspectives in comparison to the classic reactive approach. The evaluation is carried out in a set of experiments over a real testbed, which demonstrates that the proposal outperforms the classical mobile computing scheme. 

\noindent\textit{Keywords}: Smart Public Safety, Ubiquitous Computing, Video Processing, Cloud Computing