% Resumo em língua estrangeira (em inglês Abstract, em espanhol Resumen, em francês Résumé)
\begin{center}
	{\Large{\textbf{Um Sistema de Segurança em Transporte Baseado em Video Vigilância Inteligente Assistida por Computação em Nuvem para Melhorar a Assistência ao Crime em Cidades Inteligentes}}}          
\end{center}

\vspace{0.5cm}

\begin{flushright}
	Author: Hugo Barros Camboim\\
	Orientador: Prof. Augusto José Venâncio Neto, Ph.D.
\end{flushright}

\vspace{0.5cm}

\begin{center}
	\Large{\textsc{\textbf{Resumo}}}
\end{center}

\noindent Vigilância Inteligente desempenha um papel fundamental na assistência tecnológica em cenários de segurança pública através de seu potencial em detectar objetos e eventos em tempo real. Essa tecnologia tem atraído atenção especial no campo de segurança pública, particularmente em ambientes veiculares pela habilidade em assistir cenários de missão crítica através de detecção automática por vídeo em tempo real(ex. assaltos, sequestros, atos de violência, etc.). Consequentemente, é possível permitir que autoridades realizem uma compensação entre reativa e pró-ativa na tomada de ações, buscando manter sistemas de alertas customizados, realizando um planejamento mais eficiente, e prover melhor qualidade de vida à sociedade. Além disso, a associação de recursos de Computação em Nuvem é susceptível de proporcionar perspectivas avançadas para aplicações de Vigilância Inteligente, em termos de aprovisionamento reforçado: (i) processamento, para permitir implantação de técnicas de computação inteligente para melhorar a precisão; (ii) armazenamento, para adotar uma grande variedade de templates; (iii) e, acesso ubíquo para que qualquer dispositivo permaneça conectado em qualquer lugar. A investigação realizada neste trabalho é focada na abordagem de Vigilância Inteligente baseada em Computação em Nuvem conduzida para permitir execução de aplicações móveis baseadas em eventos no domínio de segurança pública. O objetivo é permitir que aplicações móveis otimizadas, as quais explorando um esquema baseado em eventos, possam prover perspectivas de baixo consumo de recursos(energia, CPU e rede) em compração à abordagem reativa clássica. Uma avaliação é feita através de um conjunto de experimentos realizados em um testbed real, os quais demonstram que o trabalho proposto supera o esquema de computação móvel clássico.

\noindent\textit{Palavras-chave}: Segurança Pública Inteligente, Computação Ubíqua, Processamento de vídeo, Computção em Nuvem.
