% Capítulo 3
\chapter{Related Work}

In traditional surveillance computing systems, specific service applications are deployed as a way to help detecting potential crime events, as well as triggering crime assistance, with more agility and accuracy than when carried out only by humans. As a consequence, these systems demand a huge amount of resources (e.g., processing, storage and networking) for affording multimedia data transmissions to remote systems that will execute on them appropriate real-time analytics. Therefore, such infrastructures are very costly, by requiring to embody high-performance and robust servers to host very specific service applications with reliable capabilities to cope to the rigorous system and scenario demands. 

The present dissertation aims at evolving the scenario described hereinabove, with the main goal to enhance safety in transportation system through exploiting emerging technologies in the research fields of ubiquitous and smart computing. We claim that the integrated use of smart surveillance techniques and cloud computing play a key role to afford the target smart transportation-safety system. In regards to smart surveillance, the goal is to produce data (video, image and/or sound) from multimedia sensor devices, and deploy specific algorithms (at the device or at remote systems) to afford automatic real-time detection of safety event. In what concerns cloud computing, this largely-used technology provides high resource (processing, storage, etc.) availability, service reliability and availability as core competitiveness and advantage, specially in mission critical use cases. 

With the perspective to efficiently achieve efficient service platform for smart transportation-safety system, we believe that following key design principles must be met:

\begin{itemize}
\item Agile and accurate detection and classification of crime events in real-time based on video surveillance devices;
\item Heavyweight service applications provisioned at the cloud, in order to afford high-performance hub-service and ubiquitous access;
\item Distributed service approach, for scalability perspectives;
\item Event-driven mobile applications, for allowing lightweight computing and energy sustainability at the mobile devices in possess by crime assist team; 
\item Raw video or sensor data storage at the cloud.
\end{itemize}

This Section lists the most relevant related works within this dissertation's field of study. Analysis over corresponding concepts, characteristics, architecture and others aspects are provided taking into account the key design principles that has being outlined above for affording the design of efficient smart safety-transportation systems. The list of related works comprises attempts that are available in both the research community and in the market.

\section{Research works}

With regards to the efforts applied by the research community in the field of smart transport-safety systems, a cloud-capable video surveillance approach is proposed in\cite{Rodriguez2012}. In this work, the main idea consists in transmitting multimedia data of video surveillance cameras, along with the linked metadata, to the Amazon's S3\footnote[9]{Amazon's S3. Available at: https://aws.amazon.com/pt/s3/. \textit{Accessed on December 8th, 2015.}} cloud service. In order to prevent video quality degradation that may occur during the data transmission to the remote cloud storage service, authors propose to carry out optimizations in the video taking into account current network conditions. Although an interesting proposal, authors rely on a questionable assumption to afford the optimizations, which means the accessibility of the required information about the current network conditions. For that, it is necessary to deploy their own network infrastructure, in order to make possible gathering all required network measurements. In third-party network infrastructures regularly used in smart city systems (e.g., cellular, telecom, cable, etc.) this measurement will be not possible by policy restrictions of the net providers, which will turn the proposal unfeasible. 

A safety system, leveraging smart video surveillance technology, is presented in \cite{salmane2015}. The work addresses to evaluate abnormal events in urban scenarios for detecting safety threats. Four fixed scenarios are studied, treated and assessed. \textit{Vehicle Stopped}, \textit{Vehicle Zigzagging}, \textit{Queuing Across the Rail} and \textit{Fall of a Pedestrian}. For each scenario, a Hidden Markov Model is developed with the aim to estimate trajectories and then detect dangerous situations at a road.  The proposal leverages the Wireless Access for Vehicular Environment (WAVE) networking technology to alarm approaching users. Although interesting, cloud computing technology is not used in the proposal architecture. 

\cite{Moniak2007} proposes a transmission system to connect urban bus safety systems and corresponding control centers. In this work, the issue in offering high data rates in mobile networks, especially in vehicular networks, are addressed. A complete transmission chain developed for a robust and high data rate wireless link is used to transmit audio and video streams between an Urban Bus and a Control Center for safety applications. A real transmission system has been built on the basis of the simulation and experiments are made in real testbed to evaluate proposed system. The work focus exclusively in affording secure data transmissions among Urban Bus and a Control Center, missing neither smart surveillance nor cloud computing.  

An agency-like prototype model for cloud-based smart surveillance applications is proposed in \cite{FANG2013}, which is based on the design of hardware monitoring with a failure-handling mechanism. The authors design a software that promises to share infrastructure in building smart surveillance applications through cloud. The main idea consists in allowing to both submit and schedule data flow (audio or video) provided by any language, using a single channel. Although intercommunication details are considered in this work, details for automated data evaluation are missing. These features are already envisioned by the \textit{IaaS} cloud Computing paradigm. Similarly, a platform based on Apache Hadoop2 is proposed in \cite{Xiang2013}, which seeks to handle overloads for video conversion/compression and storage tasks. Once video quality is an essential feature in video automated evaluation, the compression technique that is deployed for this work is not suitable for keeping evaluation algorithms working properly.

\cite{Wen2010} proposed a cloud-based framework for surveillance which sends all the collected image frames to the cloud in real time for processing. In this work, the authors design the  WAMP - \textit{Worker Agent Messaging Protocol}, a third-party communication protocol that operate amongst clients and worker agents. The WAMP is based on a \textit{request/response} messaging approach, with following similar structure: the message header contains the description of the demanding operation (e.g. calculation of the topology between two cameras); and the message body is populated with the parameters or data required to complete the target operation (e.g. the labels of two target cameras). The operational messages sent by agents are notably expensive in this approach, and overloads the system exponentially with the increasing number of monitoring points, which is undesirable for large-scale environments.

A set of Streaming-based proposals are highlighted below. In the work carried out by \cite{Cheng2014}, a software architecture for audio transmission to network cameras is presented. This work proposes audio integration between heterogeneous camera devices in order to improve object detections, as example the author cited a gunshot sound, to facilitate detection of a dangerous situation.  Moreover, \cite{Paul2013} proposes a client-server surveillance solution, in which images gathered from android device camera are transmitted towards cloud service applications for remote detection and recognition tasks. A cloud-based smart surveillance framework for robots is introduced in \cite{Yang2015}, that leverages cloud computing services in ways to extend both computing and storage capabilities of the local robots. All the data collected by the robots are saved in the cloud, assuming that all required computing resources are allocated in an on demand way. Since robots are not prepared to locally process the video data, the data must be streamed to the cloud infrastructure for video processing. 

All works described in the previous paragraph share the same key need for available broadband connectivity to allow video data transport with high quality, otherwise the accuracy in detection and recognition tasks carried out at the cloud cannot be assured. This performance issue is raised by the streaming-based principle adopted to afford multimedia data transmissions to remote systems hosting the targeted processing service application, that is in charge to detect and react on the crime event. Furthermore, streaming-based surveillance applications are subject of video quality degradation that may be imposed by the conditions of the available networking technology, in terms to instability and/or overload, that is emphasized in wireless scenarios and becomes worst in mobile ones (focus of this work). 

\section{Market-Available Solutions}
 
The smart surveillance market is becoming more and more popular, and increasing day by day its popularity and penetration in the society by the availability of low-cost cameras embodying powerful image processing algorithms. 
Usually, a home "smart" camera is able to instantaneously alerting the owners of any movement it detects,  exploiting Internet connectivity through a smartphone, computer, tablet or smart Watch via local Wi-Fi to send email and/or text message over the cellular infrastructure (if it is the case). Although very useful when the detection corresponds to on-going critical security threats (intruder), this service can become a hassle when the notifications are constantly sent for no significant reason (e.g., upon detecting an inoffensive
 butterfly or even pet). The Netatmo\footnote[10]{Netatmo. http://www.techtimes.com/articles/122385/20160108/netatmo-presence-smart-surveillance-camera-can-tell-if-it-s-an-intruder-or-your-cat-lurking-outside-your-home.htm. \textit{Accessed on November 2nd, 2016.}} Presence Smart Surveillance Camera is an example. Although providing capabilities to differentiate between a person and an animal, such smart cameras category are unsuitable for the target scenario of this work. Smart surveillance transportation-safety, quick detection of crime objects with high accuracy under unpredictable motion conditions and different noising (e.g., luminosity) is strongly required due to its mission critical characteristics. Other video surveillance service category available in the market is focused on allowing access to private security system (at home, buildings, companies, etc.) quickly and easily, avoiding complicated settings (e.g., dynamic DNS) for forwarding. 
The list of such surveillance services include the IEVDA\footnote[11]{IEVDA. https://www.iveda.com/technology/iveda-cloud-whitepaper/. \textit{Accessed on December 12th, 2015.}}, Skywatch\footnote[12]{Skywatch. http://www.marketwired.com/press-release/skywatch-inc-announces-smart-video-surveillance-cloud-service-at-cloud-expo-2015-2029404.htm. \textit{Accessed on December 12th, 2015.}} and Intelbrascloud \footnote[13]{Intelbrascloud. http://www.intelbrascloud.com.br. \textit{Accessed on December 12th, 2015.}}, and many others. On these set of application, a cloud infrastructure enables ubiquitous access to real-time content of pre-configured surveillance cameras. The cloud infrastructure is only in charge to store the video produced by the surveillance cameras, whereas users can access through web page or mobile application. Advanced video processing support, to detect security threats for instance, is missing on this set of safety applications. 

Brickcom\footnote[14]{Brickcom. http://www.brickcom.com/. \textit{Accessed on December 12th, 2015.}} provides solutions based on IP cameras. This company's approach provides a solution only for a very specific use case, and it is not suitable for generic use. Moreover,  Brickcom cloud employment allows for both storage and streaming use. In the same way, Smartvue\footnote[15]{Smartvue - Cloud video Surveillance Solutions. http://smartvue.com/index.html. \textit{Accessed on December, 12th, 2015.}} is yet another cloud-based video surveillance market solution that offers a storage service and mobile monitoring application for end users to access their private systems.

The ARUOM Safety Technology \footnote[16]{Aruom. http://www.aruom.com.br. \textit{Accessed on November 4th, 2016.}} . The company's catalog includes following main products: (i) bike patrol refers to a bicycle that can be used for crime preventive operations. The bike carries a set of surveillance subsystems, including cameras, tablet and RF link, that can be used by the police driver to capture surrounding images and call crime assist; (ii) van Renault Master, a tactical vehicle solution for Police mobile command and control unit. The vehicle  caries a set of monitoring technologies, including perimeter video cameras, TVs and RF link. The solution can be used by security forces for strategic monitoring of major events, mobilizations, urban monitoring, patrolling, and investigations with intelligence. In the scope of this dissertation, the company lacks smart technologies by relying exclusively in humans for controlling the systems.  

\section{Related Work Outcomes}

The works that are mentioned in this section are summarized below in Table 1 and classified together with their singular features over the key requirements we listed for denoting an efficient smart safety public transportation system, namely:


\begin{enumerate}
\item In-vehicle video processing for agile detection of targeted crime objects;
\item Image processing based event classification provisioned as a cloud service;
\item Instant crime event notification as a cloud service, to trigger the best crime assist through mobile application(s);
\item Event-driven operating support paradigm, for performance-enhanced and energy-efficient perspectives of application(s) running at crime assist mobile devices;
\end{enumerate}

\begin{table}[h!]
\centering
\caption{Comparison of the related work and the proposed framework}
\label{Comparison of the related work and the proposed framework}
\begin{tabular}{|l|l|l|l|l|l|}
\hline
\rowcolor[HTML]{C0C0C0} 
                             & \multicolumn{5}{l|}{\cellcolor[HTML]{C0C0C0}Features}                 \\ \hline
\rowcolor[HTML]{C0C0C0} 
Work                         & \#1 & \#2                      & \#3 & \#4 						& \#5                      \\ \hline
\cite{Rodriguez2012}     &     &                          &     &     						& \cellcolor[HTML]{C0C0C0} \\ \hline
\cite{salmane2015}      &     &                          &     &     						& \cellcolor[HTML]{C0C0C0} \\ \hline
\cite{dey2012smart}  &     &                          &     &     						& \cellcolor[HTML]{C0C0C0} \\ \hline
\cite{FANG2013}       &     & \cellcolor[HTML]{C0C0C0} &     &     						& \cellcolor[HTML]{C0C0C0} \\ \hline
\cite{Xiang2013}      &     &                          &     &     						& \cellcolor[HTML]{C0C0C0} \\ \hline
\cite{Sunehra2014} &     &                          &     &     						& \cellcolor[HTML]{C0C0C0} \\ \hline
\cite{Paul2013}    &     & \cellcolor[HTML]{C0C0C0} &     &     						&                          \\ \hline
\cite{Wen2010}       &     &                          &     &     						& \cellcolor[HTML]{C0C0C0} \\ \hline
\cite{Yang2015}      &     & \cellcolor[HTML]{C0C0C0} &     &     						& 						   \\ \hline
IEVDA				         &     & 						  &     &     						& \cellcolor[HTML]{C0C0C0} \\ \hline
Skywatch			         &     & 						  &     & \cellcolor[HTML]{C0C0C0}  & \cellcolor[HTML]{C0C0C0} \\ \hline
Brickcom			         &     & 						  &     &     						& \cellcolor[HTML]{C0C0C0} \\ \hline
Smartvue			         &     & 						  &     & \cellcolor[HTML]{C0C0C0}  & \cellcolor[HTML]{C0C0C0} \\ \hline
ARUOM	 &      &  &      &      						&\cellcolor[HTML]{C0C0C0}                          \\ \hline
Intelbrascloud	 &      &  &      &      						&\cellcolor[HTML]{C0C0C0}                          \\ \hline


\end{tabular}
\end{table}

The Table 3.1 reveals that none of the most relevant related works are able to meet the key requirements we claimed to designate an efficient smart transportation-safety system, for the reasons described in the following. First of all,  a set of smart surveillance solutions analyzed in this section is intended to provision storage and real-time video streaming access as a cloud service, thus missing intelligent capabilities. Other set of solutions, provide detection of security threat events as a cloud service based on video streaming of surveillance cameras remote installations, which is bandwidth-consuming and may not guarantee accuracy by video quality degradation of networking conditions. Moreover, the dependency in metropolitan broadband networking is a limitation and serious challenge for smart transportation-safety systems, since it may not be afforded with ubiquitous access and seamless connectivity, specially in developing countries (e.g., Brazil). Moreover, it is very costly when using third-party network operators connectivity, and the multimedia networking management is deployed in a carrier-grade way, which may also be not supported, and thus restricts the Quality of Experience (QoE) that is highly-required by the video processing applications.

All these outcomes motivate to carry out this work on designing an efficient smart safety public transportation system, which leverages the assistance of in-vehicle smart surveillance technology to afford agile detections of criminal events at buses in real-time. The in-vehicle smart surveillance system interworks with cloud image processing task, which is in charge for classifying the event that has being announced by the target in-vehicle system. Under a crime event matching confirmation, the cloud system aims at finding and triggering the best nearest crime assist mobile application. The resulting approach foresees an ecosystem with overall optimization rates, varying from low mobile network bandwidth use (and consequently cheaper), as well as processing overload and energy consumption in the end mobile devices. Lastly, but not least, the most complex tasks are deployed at the in-vehicle system (detection) as a first instance, and at the cloud system (confirmation) as a second instance. Hence, the end mobile application keeps working in a light-weighted event-driven approach for achieving both performance-enhanced and energy-sustainable perspectives.

\section{Conclusion}

In this chapter, a set of relevant related work, with varying applicability, architecture and technologies, is analyzed taking into account a list of design principles and requirements we believe that are key providing efficient smart surveillance transportation-safety systems. The limitations  highlighted in Table 1 reveal the strong relevance in carrying out this dissertation, with perspectives  of great contributions in both research community and market of the very complex and challenging smart surveillance transportation-safety systems field. Moreover, this contribution has great impact and value-addition in the quality of citizen's life at worldwide societies.

Next chapter describes in details the most relevant contribution of this dissertation, the proposal. 