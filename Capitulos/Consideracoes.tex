% Considerações finais
\chapter{Conclusion and Suggestions for Future Work}


This work proposes FISVER as a means of allowing the current Safety Transport systems to evolve towards the future smart Surveillance. The main motivation in doing this work is the sharp increase of crime events in Brazilian public transportation system. Several challenges was found due Brazil's mobile network limitations, otherwise, proposed framework proved to overcome them by adopting some strategies, there are:

\begin{itemize}
\item In-vehicle video processing for agile detection of targeted crime objects;
\item Image processing based event classification provisioned as a cloud service;
\item Instant crime event notification as a cloud service, to trigger the best crime assist through mobile application(s);
\item Event-driven operating support paradigm, for performance-enhanced and energy-efficient perspectives of application(s) running at crime assist mobile devices.
\end{itemize}

A review on relevant related works was done, in order to verify the relevance and contribution of the present proposal.  Basically, none of the related works are able to meet the key requirements that are claimed in this dissertation to designate an efficient smart transportation solution, for the reason that: (i) a set of smart surveillance solutions are intended to provision storage and real-time video streaming access as a cloud service, thus lacking smart computing capabilities; (ii) solutions envisioning to detect security threat events as a cloud service are based on streaming of remote surveillance cameras, thus denoting a severe bandwidth-consuming approach, costly and miss in guaranteeing QoE; lastly, (iii) the typical concentration on mobile devices to carry heavyweight smart computing features is not cost-efficient and severely affects the device performance and survivability.  All these outcomes motivate to carry out the present work, with great relevance in scientific research activity perspectives.

The FISVER proposal deploys a holistic cloud infrastructure that embodies a hub of modern services and applications that seek to provide a seamless, integrated Sensor and Transport Safety Application as a cloud service with ubiquitous access. The main benefit of FISVER is that it can alleviate the task of designing transport safety applications by deploying a notification-based approach to ensure an optimized performance and the survival of the mobile devices. Moreover, FISVER supports customized alerts to provide enhanced insights while keeping the Transport Safety applications lightweight. The FISVER  proposal features a modular architecture of components and subcomponents, that interwork to provision value-added smart procedures, and are accessible through open web-based interfaces. 

A set of experiments has been carried out following the premise to assess the suitability of the FISVER proposal in helping to detect security threats at public bus transportation use case in real-time, as well as agile calling appropriate crime assistance units. The performance evaluation considers prototyping the FISVER approach on a real tested, for accurate benchmarking. The methodology applied in the evaluation defines a Typical Smart Surveillance System Deployment, that is submitted to the same conditions as for the FISVER Built-in set of experiments. Each prototype configuration has been submitted to different notification load conditions, with benchmarking centered on the resulting experience at the mobile application side.  The benchmarking considered analysis in the runtime statistics collected while running  all set of experiments. 

The collected results indicate that FISVER achieves great improvements in regards to CPU load (75.6\%), bandwidth consumption (99.99\%) and ) energy efficiency (99.54\%) when compared to the behavior of the Typical Deployment set of experiments. The outstanding performance improvement of the FISVER approach over the Typical Deployment scenario in regards to CPU load (from 44.84\% to 10.94\%) is allowed through the effect of preventing the necessity to report the event-driven mobile application whenever the cloud service matches a potential crime threat. This way, FISVER reveals a significant event-constrained approach over the Typical Deployment configuration, which allows to optimize the networking performance at both upstream (from 0.2 kbps to 29.53 kbps) and downstream (from 512 kbps to 0.52 kbps) transport services. Finally, the lightweight event-driven approach that FISVER allows for mobile applications dramatically outperforms the streaming-based strategy of Typical Deployment use cases, with optimization rates from  10.61\% to 4.88\%. 

The results that are obtained in the evaluations demonstrate the suitability that the FISVER proposal takes over  the Typical Deployment testbed in affording cloud-enabled surveillance based transport-safety services. The assessment results reveals outstanding system performance and device survivability behaviors of FISVER over the Typical Deployment, enforcing its essential contribution in mission critical situations, with agility and efficiency. 

\section{Future Work}

The outcomes of this work raise needs for future work efforts in different fields of research, as in the following. The FISVER proposal is considered to be integrated on real security systems that are currently deployed at municipalities, in order to assess his suitability and performance. Moreover, FISVER is envisioned for use in other mission critical use cases, with the perspective to allow detecting security threats in urban areas (squares, streets, beaches, etc.), buildings/homes, and so on. Finally, privacy aspects are not under consideration by this dissertation, because the main focus is on the infrastructure and networking features. 

